\begin{abstract}
    The purpose of the experiment was to investigate properties of solid state materials, particularly light emitting diodes (LEDs). LEDs exploit the bandgap of semiconductor materials to emit light when voltage is applied, functioning like a reverse solar cell diode. During the experiment the threshold voltage was investigated for a white LED, in this case a blue LED with a phosphor coating. The temperature dependency and freeze-out effect was also investigated with a yellow LED, cooling it in liquid nitrogen. Furthermore, red, green, and blue LEDs where used to check if they could excite each other and produce electricity like a solar cell diode.

    The result of the experiment gave the threshold voltage of the white LED to be 2.7 V, and then a linearly increasing intensity afterwards. The yellow LED showed a decrease in current and increase in intensity as well as a shift in wavelength indicating a change in bandgap due to freeze-out. Finally, the red, green, and blue LEDs showed that the bandgap of the emitting LED must be higher than the LED that is to be excited to produce any voltage.
\end{abstract}
