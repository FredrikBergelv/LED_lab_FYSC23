\section{Discussion}
\subsection{Part 1}
The emitted light has two main peaks. One peak was in the blue part of the spectrum while the other, broader peak was at the yellow part of the spectrum. This combination appears white to our eyes. The reason for the yellow peak is that the inside of the LED is coated with phosphor which is excited by the LED and then re-emit light in the yellow part of the spectrum.

After the threshold voltage is reached the current increases linearly with the voltage. The reason is that the amount of electron-hole pairs created is proportional to the energy deposited by the voltage. The voltage need to be applied in the forward direction to shin because there is no diffusion current in the reverse direction, and it is this current that recombines to emit light.

The intensity of the light increases with increasing current and voltage, which is expected since this increases the amount of electron and thus the rate of recombination. More recombination means more light Different coloured LEDs have different bandgaps and the threshold voltage will therefore be different. For white LEDs a blue or preferably violet is preferred since this excited the phosphor and the resulting light is white.

\subsection{Part 2}
Do you see any dependency of the current on the temperature, if the same voltage is
supplied to the LED?
Does the light emission intensity changes? Why/why not?
Do you see a change in wavelength? Why/Why not?
Do you expect these behaviours for all LEDs? Why/ why not?


\subsection{Part 3}
For a diode used as a solar cell the bandgap should be small enough to absorb as much light as possible from the sun's spectrum, but large enough to extract energy. Too small of a bandgap will be easily excited but after thermalization of the election most of the energy is gone and turned to heat and cannot be used for electricity generation, while too large of a bandgap will not be excited enough by the light. For the LED's tested in the lab, the best one would be the red LED since it will be excited by the largest points of the sun's spectrum, and is still sufficiently great to extract energy from the excited electron. The other colours would absorb too small of a fraction of the sunlight to be efficient. 