\section{Discussion}
\subsection{Part 1}
The emitted light has two main peaks. One peak was in the blue part of the spectrum while the other, broader peak was at the yellow part of the spectrum. This combination appears white to our eyes. The reason for the yellow peak is that the inside of the LED is coated with phosphor which is excited by the LED and then re-emit light in the yellow part of the spectrum.

After the threshold voltage is reached the current increases linearly with the voltage. The reason is that the amount of electron-hole pairs created is proportional to the energy deposited by the voltage. The voltage need to be applied in the forward direction to shin because there is no diffusion current in the reverse direction, and it is this current that recombines to emit light.

The intensity of the light increases with increasing current and voltage, which is expected since this increases the amount of electron and thus the rate of recombination. More recombination means more light Different coloured LEDs have different bandgaps and the threshold voltage will therefore be different. For white LEDs a blue or preferably violet is preferred since this excited the phosphor and the resulting light is white.

\subsection{Part 2}
The current drops when the sample was dropped into liquid nitrogen, which is due to the fact that the resistance is correlated with the temperature. Thus, a much lower temperature gave a lower resistance, and therefore a lower current. The light emission intensity increased when submerged. This is due to low temperature results in fewer phonons and thus less phonon scattering. Hence, a low temperature means more electron- hole recombination, and thus more light emission. 

The change in wavelength of the LED corresponds to the low temperature implying a freeze out of the LED. This means that the donors/ acceptors does not have enough energy to get ionized and no doping occurs. Thus, the bandgap is increased, implying a lower wavelength of the emitted photons. This freeze out phenomenon should affect all doped materials. 


\subsection{Part 3}
For a diode used as a solar cell the bandgap should be small enough to absorb as much light as possible from the sun's spectrum, but large enough to extract energy. Too small of a bandgap will be easily excited but after thermalization of the election most of the energy is gone and turned to heat and cannot be used for electricity generation, while too large of a bandgap will not be excited enough by the light. For the LED's tested in the lab, the best one would be the red LED since it will be excited by the largest points of the sun's spectrum, and is still sufficiently great to extract energy from the excited electron. The other colours would absorb too small of a fraction of the sunlight to be efficient. 