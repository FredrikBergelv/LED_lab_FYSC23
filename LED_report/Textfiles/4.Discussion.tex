\section{Discussion}
\subsection{Part 1}
When using different LED colours, do they behave the same (if enough time is available)?
Why/why not?

The emitted light has two main peaks. One peak was in the blue part of the spectrum while the other, broader peak was at the yellow part of the spectrum. This combination appears white to our eyes. The reason for the yellow peak is that the inside of the LED is coated with phosphor which is excited by the LED and then re-emit light in the yellow part of the spectrum.

After the threshold voltage is reached the current increases linearly with the voltage. The reason is that the amount of electron-hole pairs created is proportional to the energy deposited by the voltage.

The voltage need to be applied in the forward direction to shin because there is no diffusion current in the reverse direction, and it is this current that recombines to emit light.

The intensity of the light increases with increasing current and voltage, which is expected since this increases the amount of electron and thus the rate of recombination. More recombination means more light.

Different coloured LEDs have different bandgaps and the threshold voltage will therefore be different. For white LEDs a blue or preferably violet is preferred since this excited the phosphor and the resulting light is white.



\subsection{Part 2}