\section{Introduction}
Semiconductors are becoming more and more important for our society. One important application of semiconductors is light emitting diodes (LEDs) \cite{hofmann2015} which are used in e.g. displays and lighting, but also have application in medicine. LEDs also show a much better efficiency than earlier lightbulbs, thus decreasing energy consumption \cite{hofmann2015}, something which is becoming increasingly more important. The basic part of the LED, the pn-junction, is also used in solar cells and can be used to produce energy from light \cite{hofmann2015}.

\section{Theory}
\subsection{Bandgap}
The bands in solid state materials are energy continuum that arises in solid state materials due to the number of atoms close to each other \cite{hofmann2015}. The valance band is the highest band under the Fermi energy and where the electrons are bound to the atoms. The conduction band is the lowest band above the Fermi energy, and where electrons can move freely \cite{hofmann2015}. The bandgap is the energy difference between the valance band and the conduction band and can be used to describe different materials. For conductors the bandgap is zero and electrons are free to move, for insulators the bandgap is large and a lot of energy is needed to excite an electron from the valance band to the conduction band, and for semiconductors the bandgap exist but is smaller than for insulators \cite{hofmann2015}. The definition of the size of the bandgap for insulators and semiconductors is a bit arbitrary, but is usually said to be around 3 eV \cite{hofmann2015}.

\subsection{Semiconductors}
Semiconductors, as stated in the previous section, have a bandgap under 3 eV but above 0 eV. This is a bandgap which can quite easily be excited and opens up for many applications \cite{hofmann2015}. One example is exciting the bandgap using light from the sun to produce energy. Applying a voltage to a semiconductor can also excite the bandgap. If an electron is excited from the valance band to the conduction band, a hole is left in the valence band which behave as a positive charge \cite{hofmann2015}. An excited electron will thermalize in the conduction band and lose energy to phonons and then eventually recombine with a hole in the valance band \cite{hofmann2015}.


\subsection{Doping}

\subsection{Pn-junction}

\subsection{LED}